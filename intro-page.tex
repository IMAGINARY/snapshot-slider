\documentclass[10pt]{article}
\usepackage{helvet}
\usepackage[USenglish]{babel}
\usepackage{relsize}
\usepackage{microtype}
\usepackage{xcolor,graphicx,adjustbox}
\usepackage[paperwidth=421pt,paperheight=726pt,hmargin=1.8pt,tmargin=12mm,bmargin=0.5pt]{geometry}
\usepackage{layout}
\pagestyle{empty}
\begin{document}%
%\layout
\pagecolor{black}%
\color{white}%
\relscale{1.43}%
{\centering%
    \sffamily%
    \minsizebox{\linewidth}{!}{\bfseries\Huge SNAPSHOTS}\\[0.2cm]
    \minsizebox{\linewidth}{!}{of modern mathematics from Oberwolfach}
}
\vfill

\noindent Snapshots are short, easy to understand articles on recent topics of mathematical research. They explain mathematical problems and ideas in an accessible and understandable way, and provide exciting insights into current topics of the mathematical community for everyone who is interested in modern mathematics.

The snapshots are all written by experts in their fields: At the Mathematisches Forschungsinstitut Oberwolfach, every week 50 to 60 experts from all over the world work together on current challenges in the field of mathematics. They introduce new results, discuss different approaches and develop new ideas. After each symposium, the institute asks selected participants to explain one aspect of their research in a comprehensible language to a general audience in a few pages. A team of young mathematicians revise the articles in close collaboration with the authors and assists them in communicating complicated matters to a broad audience. Together they prepare an edited version for publication.

The snapshot project is designed to promote the understanding and appreciation of modern mathematics and mathematical research in the general public world-wide.
\vfill

\noindent The snapshots team:\\[0.2ex]
\adjincludegraphics[width=\textwidth,trim={0 {.25\width} 0 {.12\width}},clip]{team.jpg}\\[-1ex]
\maxsizebox*{\linewidth}{!}{C. Cederbaum, J. Niediek, D. Kronberg, M. Firsching, S. Tokus, A. Cooper, S. Jahns.}
\newgeometry{paperwidth=421pt,paperheight=726pt,hmargin=0cm,tmargin=24.35mm,bmargin=0.5pt}%
\noindent Resources
\begin{itemize}
    \item https://imaginary.org/snapshots
    \item https://mfo.de/snapshots
\end{itemize}
\bigskip
\bigskip

\noindent Credits
\begin{itemize}
    \item Programming: Christian Stussak
    \item Text and layout: Bianca Violet and Christian Stussak
    \item Image source: Archive of the MFO
\end{itemize}
\bigskip
\bigskip

\noindent You can slide through the contents of the available snapshots by scrolling or sliding horizontally and vertically, you can email yourself a copy or print a hardcopy of each snapshot. Alternatively scan the QR code to get the direct link to a snapshots web site.
\end{document}
